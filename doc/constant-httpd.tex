\documentclass{book}
\usepackage{sstdef}
\title{shttpd}
\begin{document}
\section{The \cmd{constant-httpd} program}

\subsection{Interface}
\begin{code}
  constant-httpd \var{file}
\end{code}
where \cvar{file} is a file name.

\cmd{constant-httpd} reads an HTTP request message from standard input,
and prints \cvar{file} in response.

Before reading any input, \cmd{constant-httpd} changes the working
directory to that named in the \cmd{\$ROOT} environment variable,
performs \cmd{chroot} to the current directory, and then sets its
group id and user id to the numeric values given in the environment
variables \cmd{\$GID} and \cmd{\$UID}, typically set with
\href{\cmd{envdir}}{http://cr.yp.to/daemontools/envdir.html}.
If it cannot carry out these operations, \cmd{constant-httpd}
complains to standard output and exits 111.  If the \cvar{file}
argument is missing it silently exits 100.

If it encounters an error in reading \cvar{file}, \cmd{constant-httpd}
exits 23.  If it encounters another error processing a request,
\cmd{constant-httpd} exits 21.  Otherwise, it exits 0.

\cmd{constant-httpd} accepts HTTP/0.9, HTTP/1.0, and HTTP/1.1
requests.  For a request with the host name \cvar{host},
\cmd{constant-httpd} looks for \cmd{./\var{host}/\var{file}}, after
replacing \cmd{/.} with \cmd{/:} and \cmd{//} with \cmd{/} in
\cmd{./\var{host}/\var{file}}.  For a valid request lacking a host name,
\cmd{constant-httpd} uses \cmd{0} as the host name.

If the environment variable \cmd{\$REQUESTHOST} is set, then
\cmd{constant-httpd} uses its value as \cvar{host}, overriding any host given in
the request.
\end{document}
