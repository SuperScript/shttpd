%& sstdoc
\banner{shttpd}
\begin{document}
\image{../sstlogo.gif}{sstlogo}
\parent{../superscript}{\SST}
\parent{../software}{Software}
\parent{intro}{shttpd}
\chapter{The \cmd{cgi-httpd} program}

\section{Interface}
\begin{code}
  cgi-httpd \arg{prog}
\end{code}
where \carg{prog} is one or more arguments specifying a program to run
for each valid request.

\cmd{cgi-httpd} reads an HTTP request message from standard input,
and treats the URI path as the name of a CGI script.  The script is
expected to print its reply to standard output.

Before reading any input, \cmd{cgi-httpd} changes the working directory to that
named in the \cmd{\$ROOT} environment variable, performs \cmd{chroot} to the
current directory, and then sets its group id and user id to the numeric values
given in the environment variables \cmd{\$GID} and \cmd{\$UID}, typically set
with \href{\cmd{envdir}}{http://cr.yp.to/daemontools/envdir.html}.  Before
executing \carg{prog}, \cmd{cgi-httpd} sets the execution path to \cmd{/bin}.
If it cannot carry out these operations, \cmd{cgi-httpd} complains to standard
output and exits 111.

If it encounters an error in processing a request, \cmd{cgi-httpd}
exits 21.  Otherwise, it returns the exit code of \carg{prog}.

For a request specifying the hostname \carg{H}, \cmd{cgi-httpd}
changes its working directory to \cmd{./host/\carg{H}} after replacing
\cmd{/.}  with \cmd{/:} and \cmd{//} with \cmd{/} in
\cmd{./host/\carg{H}}.  When a valid request lacks a host name,
\cmd{cgi-httpd} uses \cmd{0} as the host name.  After changing the
working directory, \cmd{cgi-httpd} sets the cgi environment variables
and executes \carg{prog}.

\cmd{cgi-httpd} accepts HTTP/0.9, HTTP/1.0, and HTTP/1.1 requests.  It
supports HEAD, GET, and POST, and rejects all other methods.

\cmd{cgi-httpd} manipulates the following environment variables:

\cmd{\$AUTH_TYPE} is unset.

\cmd{\$CONTENT_LENGTH} is the content length of the request, if
applicable, or unset.

\cmd{\$CONTENT_TYPE} is the content type of the request, if
applicable, or unset.

\cmd{\$GATEWAY_INTERFACE} is the string \cmd{CGI/1.1}

\cmd{\$PATH_INFO} is the path portion of the request URI, after URL
decoding.

\cmd{\$PATH_TRANSLATED} is unset.

\cmd{\$QUERY_STRING} is the query string portion of the request URI.

\cmd{\$REMOTE_ADDR} is set to the value of the environment variable \cmd{\$TCPREMOTEIP}.

\cmd{\$REMOTE_HOST} is set to the value of the environment variable \cmd{\$TCPREMOTEHOST}.

\cmd{\$REMOTE_USER} is set to the value of the environment variable \cmd{\$TCPREMOTEINFO}.

\cmd{\$REQUEST_METHOD} is set to the method of the request URI.

\cmd{\$SCRIPT_NAME} is the empty string.

\cmd{\$SERVER_NAME} is the name of the host given in the request, if
applicable, and 0 otherwise.

\cmd{\$SERVER_PORT} is the value of the environment variable \cmd{\$TCPLOCALPORT}.

\cmd{\$SERVER_PROTOCOL} is one of the strings \cmd{HTTP/0.9},
\cmd{HTTP/1.0}, or \cmd{HTTP/1.1}.

\cmd{\$SERVER_SOFTWARE} is the string \cmd{cgi-httpd}.


\section{Examples}
The \cmd{cgi-example} program running under \cmd{cgi-httpd} provides
an example of a single-address, single-form cgi server.  These
instructions assume that the user accounts \cmd{shttp} and
\cmd{shttplog} exist, and that shttpd was installed in
/usr/local/shttpd.

Create a directory to house shttpd service directories, for example:
\begin{code}
  mkdir /usr/local/shttpd/service
\end{code}

Configure a \cmd{cgi-example} server to listen on a port of your choice:
\begin{code}
  cgi-config shttp shttplog /usr/local/shttpd/service/cgi-example \arg{port} cgi-example
\end{code}

Copy \cmd{cgi-example} to
\cmd{/usr/local/shttpd/service/cgi-example/root/bin}.

Download and compile the \href{miscellanea}{../miscellanea/intro.html}
package.  Copy the executables \cmd{foldlines}, \cmd{revline},
\cmd{rot13}, and \cmd{wrapcr} to
\cmd{/usr/local/shttpd/service/cgi-example/root/bin}.

For each \carg{hostname} you wish accept in requests, create the
directory
\cmd{/usr/local/shttpd/service/cgi-example/root/host/\arg{hostname}}.

Finally, start the service under \cmd{svscan}:
\begin{code}
  ln -s /usr/local/shttpd/service/cgi-example /service
\end{code}

You can test the cgi-example server
\href{here}{http://www.superscript.com:30000}.

Documentation for the programs used by the example server is
\href{here}{../miscellanea/intro.html}.
\end{document}
