%& sstdoc
\banner{shttpd}
\begin{document}
\image{../sstlogo.gif}{sstlogo}
\parent{../superscript}{\SST}
\parent{../software}{Software}
\parent{intro}{shttpd}
\chapter{The \cmd{constant-config} program}

\section{Interface}
\begin{code}
  constant-config \arg{user} \arg{loguser} \arg{dir} \arg{port} \carg{file}
\end{code}
where \carg{user} and \carg{loguser} are system account names,
\carg{dir} is an absolute path name, \carg{port} is a TCP port, and
\carg{file} is a file name.  The port may be specified as a name from
\cmd{/etc/services} or a numeric value.

\cmd{constant-config} creates the directory \carg{dir} as an
\href{svscan}{http://cr.yp.to/daemontools/svscan.html}-format service
directory, configured to run \cmd{constant-httpd} on the TCP port
\carg{port}, running \cmd{constant-httpd} as \carg{user} and creating log
files as \carg{loguser}.

\cmd{constant-config} creates the directory \cmd{\arg{dir}/env}
containing files that establish runtime environment variables for
\cmd{constant-httpd} via \href{envdir}{http://cr.yp.to/daemontools/envdir.html}.
The variables include \cmd{\$UID}, \cmd{\$GID}, \cmd{\$ROOT}, and
\cmd{\$REQUESTHOST}.

\cmd{constant-config} creates the directory \cmd{\arg{dir}/root} for
\cmd{constant-httpd} to use as its root directory.

\cmd{constant-config} creates an empty
\cmd{\arg{dir}/env/REQUESTHOST}, causing \cmd{constant-httpd} to
respect the host named in each request.  To force
\cmd{constant-httpd} to use a particular host name, overriding the
host obtained from a request, edit \cmd{\arg{dir}/env/REQUESTHOST}
after running \cmd{constant-config}.
\end{document}
