\documentclass{book}
\usepackage{sstdef}
\title{shttpd}
\begin{document}
\section{The \cmd{constant-config} program}

\subsection{Interface}
\begin{code}
  constant-config \var{user} \var{loguser} \var{dir} \var{port} \cvar{file}
\end{code}
where \cvar{user} and \cvar{loguser} are system account names,
\cvar{dir} is an absolute path name, \cvar{port} is a TCP port, and
\cvar{file} is a file name.  The port may be specified as a name from
\cmd{/etc/services} or a numeric value.

\cmd{constant-config} creates the directory \cvar{dir} as an
\href{svscan}{http://cr.yp.to/daemontools/svscan.html}-format service
directory, configured to run \cmd{constant-httpd} on the TCP port
\cvar{port}, running \cmd{constant-httpd} as \cvar{user} and creating log
files as \cvar{loguser}.

\cmd{constant-config} creates the directory \cmd{\var{dir}/env}
containing files that establish runtime environment variables for
\cmd{constant-httpd} via \href{envdir}{http://cr.yp.to/daemontools/envdir.html}.
The variables include \cmd{\$UID}, \cmd{\$GID}, \cmd{\$ROOT}, and
\cmd{\$REQUESTHOST}.

\cmd{constant-config} creates the directory \cmd{\var{dir}/root} for
\cmd{constant-httpd} to use as its root directory.

\cmd{constant-config} creates an empty
\cmd{\var{dir}/env/REQUESTHOST}, causing \cmd{constant-httpd} to
respect the host named in each request.  To force
\cmd{constant-httpd} to use a particular host name, overriding the
host obtained from a request, edit \cmd{\var{dir}/env/REQUESTHOST}
after running \cmd{constant-config}.
\end{document}
