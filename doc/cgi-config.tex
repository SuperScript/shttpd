\documentclass{book}
\usepackage{sstdef}
\title{shttpd}
\begin{document}
\section{The \cmd{cgi-config} program}

\subsection{Interface}
\begin{code}
  cgi-config \var{user} \var{loguser} \var{dir} \var{port} \var{prog}
\end{code}
where \cvar{user} and \cvar{loguser} are system account names,
\cvar{dir} is an absolute path name, and \cvar{port} is a TCP port,
and \cvar{prog} is one or more arguments specifying a program to run
for each valid cgi request.  The port may be specified as a name from
\cmd{/etc/services} or a numeric value.

\cmd{cgi-config} creates the directory \cvar{dir} as an
\href{svscan}{http://cr.yp.to/daemontools/svscan.html}-format service
directory, configured to run \cmd{cgi-httpd} on the TCP port
\cvar{port}, running \cmd{cgi-httpd} as \cvar{user} and creating log
files as \cvar{loguser}, and executing \cvar{prog} for each cgi
request it receives.

\cmd{cgi-config} creates the directory \cmd{\var{dir}/env}
containing files that establish runtime environment variables for
\cmd{cgi-httpd} via \href{envdir}{http://cr.yp.to/daemontools/envdir.html}.

\cmd{cgi-config} creates the directory \cmd{\var{dir}/root} for
\cmd{cgi-httpd} to use as its root directory.  \cmd{cgi-config} also
creates \cmd{\var{dir}/root/bin} to hold cgi executable files, and
\cmd{\var{dir}/root/host} to hold the per-host working directories.

\end{document}
