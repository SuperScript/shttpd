%& sstdoc
\banner{shttpd}
\begin{document}
\image{../sstlogo.gif}{sstlogo}
\parent{../superscript}{\SST}
\parent{../software}{Software}
\parent{intro}{shttpd}
\chapter{The \cmd{echo-httpd} program}

\section{Interface}
\begin{code}
  echo-httpd
\end{code}

\cmd{echo-httpd} reads an HTTP request message from standard input,
adds a header, and prints the result to standard output.

Before reading any input, \cmd{echo-httpd} changes the working
directory to that named in the \cmd{\$ROOT} environment variable,
performs \cmd{chroot} to the current directory, and then sets its
group id and user id to the numeric values given in the environment
variables \cmd{\$GID} and \cmd{\$UID}, typically set with
\href{\cmd{envdir}}{http://cr.yp.to/daemontools/envdir.html}.
If it cannot carry out these operations, \cmd{echo-httpd} complains to
standard output and exits 111.  \cmd{echo-httpd} opens no files.

If it encounters a temporary error in processing a request,
\cmd{echo-httpd} exits 21.  Otherwise, it exits 0.

\cmd{echo-httpd} is a debugging tool, and not a proper http daemon.
It does not parse the request message, and handles any request, no
matter how severely deformed.
\end{document}
