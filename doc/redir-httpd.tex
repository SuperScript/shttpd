\documentclass{book}
\usepackage{sstdef}
\title{shttpd}
\begin{document}
\section{The \cmd{redir-httpd} program}

\subsection{Interface}
\begin{code}
  redir-httpd
\end{code}

\cmd{redir-httpd} reads an HTTP request message from standard input,
looks up the URI in a cdb-format database, and prints a redirection
command or error message to standard output.

Before reading any input, \cmd{redir-httpd} changes the working
directory to that named in the \cmd{\$ROOT} environment variable,
performs \cmd{chroot} to the current directory, and then sets its
group id and user id to the numeric values given in the environment
variables \cmd{\$GID} and \cmd{\$UID}, typically set with
\href{\cmd{envdir}}{http://cr.yp.to/daemontools/envdir.html}.
If it cannot carry out these operations, \cmd{redir-httpd} complains
to standard error and exits~111.

If it encounters an error in processing a request, \cmd{redir-httpd}
exits~21.  Otherwise, it exits~0.

The \cmd{redir-httpd} program accepts HTTP/0.9, HTTP/1.0, and HTTP/1.1 requests.
For a request with the host name \cvar{host}, \cmd{redir-httpd} replaces
\cmd{/.} with \cmd{/:} and \cmd{//} with \cmd{/} in \cmd{./\cvar{host}}, and
looks for redirection commands in the file \cmd{./\cvar{host}/data.cdb}, created
by \href{\cmd{redir-data}}{redir-data.html}.  For a valid request lacking a host
name, \cmd{redir-httpd} uses~\cmd{0} as the host name.

\subsection{Request path parsing}

\cmd{redir-httpd} replaces \cmd{/.} with \cmd{/:} and \cmd{//} with \cmd{/} in
the request path before looking for a match.

Starting immediately to the left of any query string in the request path,
and moving from right to left,
\cmd{redir-httpd}
splits the request path at each \cmd{/} character to obtain
\cmd{\cvar{head}/\cvar{tail}},
and and attempts a cdb lookup using
\cmd{\cvar{head}/}
as the key.
Upon success,
\cmd{redir-httpd}
replaces
\cmd{\cvar{head}/}
with the lookup result, appends
\cvar{tail},
and redirects the client to the resultant URL.
When
\cmd{redir-httpd}
receives a request with an empty path, it uses
\cmd{/}
instead.
If the full request path (sans query string) ends with a
character other than
\cmd{/},
then
\cmd{redir-httpd}
first attempts a lookup using the full request path as the key.

\subsection{Logging}

When it processes a request without error, \cmd{redir-httpd} prints to
standard error a line with the form
\begin{code}
  \var{host ref match key value}
\end{code}
The string \cvar{host} is the request host name.
The string \cvar{ref} is the value of the first \cmd{Referer:} header,
  or \cmd{0} if there is no such header.
The string \cvar{match} is \cmd{unknown} if no match is found,
  \cmd{partial} for a partial path match,
  or \cmd{exact} for an exact path match.
The string \cvar{key} is the portion of the request path that matched a
  redirection command,
  or the full request path when \cvar{match} is \cmd{unknown}.
The string \cvar{value} is the replacement for \cvar{key} in the relevant
  redirection command,
  or the empty string when \cvar{match} is \cmd{unknown}.

Within each component of the log line, \cmd{redir-httpd} replaces control
characters with '\cmd{?}' and spaces with '\cmd{_}'.  If a component exceeds 100
characters in length, \cmd{redir-httpd} truncates it to 100 characters and
appends ellipses.

\subsection{Redirection examples}
Commands specifying the following substitutions
\begin{code}
  /shttpd  --> http://www.superscript.com/shttpd/intro.html
  /shttpd/ --> http://www.superscript.com/shttpd/
\end{code}
result in the following redirections:
\begin{code}
  /shttpd --> http://www.superscript.com/shttpd/intro.html
  /shttpd/cgi-httpd.html --> http://www.superscript.com/shttpd/cgi-httpd.html
\end{code}

\end{document}
