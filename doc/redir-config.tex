\documentclass{book}
\usepackage{sstdef}
\title{shttpd}
\begin{document}
\section{The \cmd{redir-config} program}

\subsection{Interface}
\begin{code}
  redir-config \var{user} \var{loguser} \var{dir} \var{port}
\end{code}
where \cvar{user} and \cvar{loguser} are system account names,
\cvar{dir} is an absolute path name, and \cvar{port} is a TCP port.
The port may be specified as a name from \cmd{/etc/services} or a
numeric value.

\cmd{redir-config} creates the directory \cvar{dir} as an
\href{svscan}{http://cr.yp.to/daemontools/svscan.html}-format service
directory, configured to run \cmd{redir-httpd} on the TCP port
\cvar{port}, running \cmd{redir-httpd} as \cvar{user} and creating log
files as \cvar{loguser}.

\cmd{redir-config} creates the directory \cmd{\var{dir}/env}
containing files that establish runtime environment variables for
\cmd{redir-httpd} via \href{envdir}{http://cr.yp.to/daemontools/envdir.html}.

\cmd{redir-config} creates the directory \cmd{\var{dir}/root} for
\cmd{redir-httpd} to use as its root directory.

\cmd{redir-config} creates \cmd{\var{dir}/root/Makefile} for compiling
\cmd{\var{dir}/root/data.cdb} with \href{\cmd{redir-data}}{redir-data.html}.


\end{document}
